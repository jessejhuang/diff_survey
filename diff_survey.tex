\documentclass[conference]{IEEEtran}
\IEEEoverridecommandlockouts
% The preceding line is only needed to identify funding in the first footnote. If that is unneeded, please comment it out.
\usepackage{cite}
\usepackage{amsmath,amssymb,amsfonts}
% \usepackage{algorithmic}
\usepackage{graphicx}
\usepackage{textcomp}
\usepackage{xcolor}
\def\BibTeX{{\rm B\kern-.05em{\sc i\kern-.025em b}\kern-.08em
    T\kern-.1667em\lower.7ex\hbox{E}\kern-.125emX}}
\begin{document}

\title{A Survey on the Role of Individual Differences on Visual
Analytics Interactions\\
\thanks{
    Boeing: Integrated Computational and Cognitive Workflows for
Improved Security and Usability \newline
DOD: Investigating the Role of Individual Differences in Visual Analytic Workflows
}
}

\author{\IEEEauthorblockN{Jesse Huang}
\IEEEauthorblockA{
\textit{Washington University in St. Louis}\\
jessehuang@wustl.edu}
\and
\IEEEauthorblockN{Alvitta Ottley}
\IEEEauthorblockA{
\textit{Washington University in St. Louis}\\
alvitta@wustl.edu}

}

\maketitle

\begin{abstract}
There is ample evidence in the visualization community that individual differences
matter. These prior works highlight various traits and cognitive abilities that
can modulate the use of the visualization systems and demonstrate a measurable
influence on speed, accuracy, process, and attention. Perhaps the most
important implication of this body of work is that we can use individual differences
as a mechanism for estimating people’s potential to effectively leverage visual
interfaces or to identify those people who may struggle. As visual literacy and
data fluency continue to become essential skills for our everyday lives, we
must embrace the growing need to understand the factors that divide our
society, and identify concrete steps for bridging this gap. This paper presents
the current understanding of how individual differences interact with
visualization use and draws from recent research in the Visualization,
Human-Computer Interaction, and Psychology communities. We focus on the
specific designs and tasks for which there is concrete evidence of performance
divergence due to individual characteristics. The purpose of this paper is to
underscore the need to consider individual differences when designing and
evaluating visualization systems and to call attention to this critical future
research direction.
\end{abstract}

\begin{IEEEkeywords}
visualization, individual differences, personality,
locus of control, cognitive abilities
\end{IEEEkeywords}

\section{Introduction}\label{Intro}
A substantial amount of work has been done to uncover the
effects that individual differences have on visualization
interactions. The implications of this research are attractive
and far-reaching; with a profound understanding of this
relationship, we could design visualizations catered to a
person’s specific needs, and infer a person’s traits from
their interactions with electronic media.  However, there is
still much work to be done, the nature of which is not always
immediately clear. 

The task of surveying the role of individual differences on
visual analytics interactions is a very open-ended problem.
There are many ways to taxonomize this research, such as by
by individual trait or experimental design. However, the
distribution of research effort is uneven; there is
far more research focusing on certain individual traits,
such as locus of control, than others, such as openness.
As such, a taxonomy that simply enumerates the work done
for each trait would be limited. Instead, we use the
well-understood traits to contextualize the
state of research for all other traits
to gain clearer insignt into productive avenues for future
research.

The well understood traits that we examine are locus of control,
perceptural speed, visual working memory, verbal working
memory. We refer to these latter three traits under the
collective umbrella of "cognitive traits", because a significant
amount of research has evaluated them together. By tracking
the progression of research for these two groups, and discussing
their roles in visual analytics interactions, we can more
accurately discuss the states of other individual differences. 
The other traits we consider include the Big Five personality traits
of Openness, Conscientiousness, Extraversion, Agreeableness,
and Neuroticism.
% talk about how Big 5 are often measured in aggregate, but studies
% rarely find significant results for each one.

\section{Well-Understood Traits}\label{WellStudied}
It is worth noting that we define these traits as "well understood"
relative to the collective body of individual traits. This
definition does not necessarily imply that there has been greater
research interest in these traits. Rather, studies often evaluate
multiple traits in conjunction and the traits we label as
"well understood" reflect greater results.
\subsection{Locus of Control}\label{LocusOfControl}
Locus of Control, the degree to which a person believes they
have control over their lives, has been studied closely
within the context of visualization. Green et al's 
"Using Personality Factors to Predict Interface Learning Performance"
was a seminal work in this effort. In this study,
participants were asked to complete personality inventories
to assess their degrees of locus of control, anxiety, neuroticism, 
extraversion, self-regulation, and tolerance of ambigiuty. They then
familiarized themselves with two real-world visualizations of the same dataset:
a row-based representaiton, and a pictoral, container-based representaiton.
Particpants then used these visualizations to answer a series of
procedural and inferential tasks, and their answers were analyzed with
respect to each measured personality trait. Significant correlations
were found for the traits of locus of control and anxiety, specifically that
those with an external locus of control tended to complete inferential tasks
more quickly, and those with greater anxiety measures tended to complete
iterative search tasks more quickly\cite{Green2010UsingPF}. These results suggested that the effects
of individual traits on visualization interactions varied with the manner and
purpose of these interactions. 

Since Green et al's study, Locus of control, and
individual differences in general have been studied with greater attention
to the precise nature of given visualizations and tasks. Ziemkiewicz et al's 
2011 study, "How locus of control influences compatibility with visualization style" assessed both procedural and inferential tasks, while
examined individual differences with a more isolated assessment of layout.
This study evaluated locus of control with four visualizations of increasingly
explicit containment metaphors. These metaphors ranged from simple indentations
to mark hierarchy to series of nested boxes. It found that those with external
locus of control answered questions on the most and least explicit metaphors
faster, and performed better than those with internal locus of control in
general. There were no response time differences observed for those with
average locus of control. Search tasks also produced no significant differences
in response time across visualizations and participants\cite{Ziemkiewicz}.
This study was a progression on Green's because it isolated visualization layout
in its study of individual traits, while reinforcing the idea that the types of
tasks performed (eg procedural vs inferential) affect interactions nonuniformly.

 At this point, there was substantive evidence that locus of control impacted
 visualization interactions with respect to layout structure
and tasks performed. Ottley et al's 2012 study, "Priming Locus of Control
to Affect Performance", extended this research by examining if Locus of Control
could be "primed" before visualization interaction. People were asked
to recall situations in which they felt in control or out of control, before
participanting in a replication of Ziemkiewicz's 2011 study. Ottley et al found
that peoples' Locus of Control could be "pushed" externally or internally, which
was reflected in the replicated study's results \cite{OttleyPriming}.

% Why LOC studies are a useful metric:
%     diverse range of tasks and visualizations assessed
%     Studies vary in:
%         Measurement (eye tracking, completion time, accuracy)
%         Visualization type (isolated layout, general)
%         Task (procedural (ie search), inferential)
% What effects do LOC have on visualization interaction?

% potential shortcomings of these papers as well
% next steps: based on what we've seen, what are possible future directions?
%  as we think about how computers can help users, perhaps openness is a trait
% that people can look at

% Openness
% - Visual metaphors show lack of significance
% - Gajos tests need for cognition: openness is highly correlated with
% need for cognition, and so further research should explore it

% Agreeableness
% - Could not find significant results
% - Most tasks shown do not have an intuitive connection to agreeableness;
% therefore this absence of significance makes sense

% identify and find behaviors showed different performances than
% inferential tasks- suggesting that the impact of traits can be very
% different depending on the task that is asked.
% since this study, a variety of different tasks across different studies
% have been evaluated.
% However, if you look at tasks requested, there is no intuitive connection
% with agreeableness, so this is why that makes sense


% - Also consider that Locus of control is the focus of many
% papers, whereas many Big 5 traits are examined as a whole.
% There is less detail given to any one of them. d
% - With locus of control section, we talk about the
% progression in terms of the different tasks that are asked
% in each study, and how they illuminate more about LOC

\subsection{Subsection}

\cite{GreenTowardsPersonalEquation}
\cite{Ziemkiewicz,ZiemkiewiczOttley,OttleyPriming,Waldo}
\bibliographystyle{ieee}
\bibliography{refs}
\vspace{12pt}

\end{document}
