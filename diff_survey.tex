\documentclass[conference]{IEEEtran}
\IEEEoverridecommandlockouts
% The preceding line is only needed to identify funding in the first footnote. If that is unneeded, please comment it out.
\usepackage{cite}
\usepackage{amsmath,amssymb,amsfonts}
% \usepackage{algorithmic}
\usepackage{graphicx}
\usepackage{textcomp}
\usepackage{xcolor}
\def\BibTeX{{\rm B\kern-.05em{\sc i\kern-.025em b}\kern-.08em
    T\kern-.1667em\lower.7ex\hbox{E}\kern-.125emX}}
\begin{document}

\title{A Survey on the Role of Individual Differences on Visual
Analytics Interactions\\
\thanks{
    Boeing: Integrated Computational and Cognitive Workflows for
Improved Security and Usability \newline
DOD: Investigating the Role of Individual Differences in Visual Analytic Workflows
}
}

\author{\IEEEauthorblockN{Jesse Huang}
\IEEEauthorblockA{
\textit{Washington University in St. Louis}\\
jessehuang@wustl.edu}
\and
\IEEEauthorblockN{Alvitta Ottley}
\IEEEauthorblockA{
\textit{Washington University in St. Louis}\\
alvitta@wustl.edu}

}

\maketitle

\begin{abstract}
This document is a model and instructions for \LaTeX.
This and the IEEEtran.cls file define the components of your paper [title, text, heads, etc.]. *CRITICAL: Do Not Use Symbols, Special Characters, Footnotes, 
or Math in Paper Title or Abstract.
\end{abstract}

\begin{IEEEkeywords}
visualization, individual differences, personality,
locus of control, cognitive abilities
\end{IEEEkeywords}

\section{Introduction}\label{Intro}
A substantial amount of work has been done to uncover the
effects that individual differences have on visualization
interactions. The implications of this research are attractive
and far-reaching; with a profound understanding of this
relationship, we could design visualizations catered to a
person’s specific needs, and infer a person’s traits from
their interactions with electronic media.  However, there is
still much work to be done, the nature of which is not always
immediately clear. This paper is a survey of the work that has
been done, and a proposal for future avenues of research.

\subsection{Challenges}\label{IntroChallenges}
The task of surveying the role of individual differences on
visual analytics interactions is a very open-ended problem.
There are many ways to taxonomize this research, such as by
by individual trait or experimental design. However, the
distribution of research effort is uneven; there is
far more research focusing on certain individual differences,
such as locus of control, than others, such as openness.
As such, a taxonomy that simply enumerates the work done
for each trait would be limited. Instead, we


\section{Locus of Control}\label{LocusOfControl}
Locus of Control, the degree to which a person believes they
have control over their lives, has been studied closely
within the context of visualization. By examining
the progression of this research, we uncover a logical
progression

- Also consider that Locus of control is the focus of many
papers, whereas many Big 5 traits are examined as a whole.
There is less detail given to any one of them. d




\subsection{Subsection}

\cite{GreenTowardsPersonalEquation}
\cite{Ziemkiewicz,ZiemkiewiczOttley,OttleyPriming,Waldo}
\bibliographystyle{ieee}
\bibliography{refs}
\vspace{12pt}

\end{document}
