\documentclass[conference]{IEEEtran}
\IEEEoverridecommandlockouts
% The preceding line is only needed to identify funding in the first footnote. If that is unneeded, please comment it out.
\usepackage{cite}
\usepackage{amsmath,amssymb,amsfonts}
% \usepackage{algorithmic}
\usepackage{graphicx}
\usepackage{textcomp}
\usepackage{xcolor}
\def\BibTeX{{\rm B\kern-.05em{\sc i\kern-.025em b}\kern-.08em
    T\kern-.1667em\lower.7ex\hbox{E}\kern-.125emX}}
\begin{document}

\title{A Survey on the Role of Individual Differences on Visual
Analytics Interactions\\
\thanks{
    Boeing: Integrated Computational and Cognitive Workflows for
Improved Security and Usability \newline
DOD: Investigating the Role of Individual Differences in Visual Analytic Workflows
}
}

\author{\IEEEauthorblockN{Jesse Huang}
\IEEEauthorblockA{
\textit{Washington University in St. Louis}\\
jessehuang@wustl.edu}
\and
\IEEEauthorblockN{Alvitta Ottley}
\IEEEauthorblockA{
\textit{Washington University in St. Louis}\\
alvitta@wustl.edu}

}

\maketitle

\begin{abstract}
This document is a model and instructions for \LaTeX.
This and the IEEEtran.cls file define the components of your paper [title, text, heads, etc.]. *CRITICAL: Do Not Use Symbols, Special Characters, Footnotes, 
or Math in Paper Title or Abstract.
\end{abstract}

\begin{IEEEkeywords}
visualization, individual differences, personality,
locus of control, cognitive abilities
\end{IEEEkeywords}

\section{Introduction}\label{Intro}
A substantial amount of work has been done to uncover the
effects that individual differences have on visualization
interactions. The implications of this research are attractive
and far-reaching; with a profound understanding of this
relationship, we could design visualizations catered to a
person’s specific needs, and infer a person’s traits from
their interactions with electronic media.  However, there is
still much work to be done, the nature of which is not always
immediately clear. 

The task of surveying the role of individual differences on
visual analytics interactions is a very open-ended problem.
There are many ways to taxonomize this research, such as by
by individual trait or experimental design. However, the
distribution of research effort is uneven; there is
far more research focusing on certain individual traits,
such as locus of control, than others, such as openness.
As such, a taxonomy that simply enumerates the work done
for each trait would be limited. Instead, we use the
well-understood traits to contextualize the
state of research for all other traits
to gain clearer insignt into productive avenues for future
research.

The well understood traits that we examine are locus of control,
perceptural speed, visual working memory, verbal working
memory. We refer to these latter three traits under the
collective umbrella of "cognitive traits", because a significant
amount of research has evaluated them together. By tracking
the progression of research for these two groups, and discussing
their roles in visual analytics interactions, we can more
accurately discuss the states of other individual differences. 
% In turn- % talk about less studied traits?

\section{Well-Understood Traits}\label{WellStudied}
It is worth noting that we define these traits as "well understood"
relative to the collective body of individual traits. This
definition does not necessarily imply that there has been greater
research interest in these traits. Rather, studies often evaluate
multiple traits in conjunction and the traits we label as
"well understood" reflect greater results.
\subsection{Locus of Control}\label{LocusOfControl}
Locus of Control, the degree to which a person believes they
have control over their lives, has been studied closely
within the context of visualization. Green and Fisher's 
"Towards the Personal Equation of Interaction:
The Impact of Personality Factors on Visual Analytics Interface
Interaction" was a seminal work in this effort. In this study,
participants were asked to complete personality inventories
to assess their degrees of locus of control, anxiety, and other
  


% - Also consider that Locus of control is the focus of many
% papers, whereas many Big 5 traits are examined as a whole.
% There is less detail given to any one of them. d
% - With locus of control section, we talk about the
% progression in terms of the different tasks that are asked
% in each study, and how they illuminate more about LOC




\subsection{Subsection}

\cite{GreenTowardsPersonalEquation}
\cite{Ziemkiewicz,ZiemkiewiczOttley,OttleyPriming,Waldo}
\bibliographystyle{ieee}
\bibliography{refs}
\vspace{12pt}

\end{document}
