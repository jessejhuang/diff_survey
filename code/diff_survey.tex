\documentclass[conference]{IEEEtran}
\IEEEoverridecommandlockouts
% The preceding line is only needed to identify funding in the first footnote. If that is unneeded, please comment it out.
\usepackage{cite}
\usepackage{amsmath,amssymb,amsfonts}
% \usepackage{algorithmic}
\usepackage{graphicx}
\usepackage{textcomp}
\usepackage{xcolor}
\usepackage{hyperref}
\def\BibTeX{{\rm B\kern-.05em{\sc i\kern-.025em b}\kern-.08em
    T\kern-.1667em\lower.7ex\hbox{E}\kern-.125emX}}
\begin{document}

\title{A Survey on the Role of Individual Differences on Visual
Analytics Interactions\\
\thanks{
    Boeing: Integrated Computational and Cognitive Workflows for
Improved Security and Usability \newline
DOD: Investigating the Role of Individual Differences in Visual Analytic Workflows
}
}

\author{\IEEEauthorblockN{Jesse Huang}
\IEEEauthorblockA{
\textit{Washington University in St. Louis}\\
jessehuang@wustl.edu}
\and
\IEEEauthorblockN{Alvitta Ottley}
\IEEEauthorblockA{
\textit{Washington University in St. Louis}\\
alvitta@wustl.edu}

}

\maketitle

\begin{abstract}
There is ample evidence in the visualization community that individual differences
matter. These prior works highlight various traits and cognitive abilities that
can modulate the use of the visualization systems and demonstrate a measurable
influence on speed, accuracy, process, and attention. Perhaps the most
important implication of this body of work is that we can use individual differences
as a mechanism for estimating people’s potential to effectively leverage visual
interfaces or to identify those people who may struggle. As visual literacy and
data fluency continue to become essential skills for our everyday lives, we
must embrace the growing need to understand the factors that divide our
society, and identify concrete steps for bridging this gap. This paper presents
the current understanding of how individual differences interact with
visualization use and draws from recent research in the Visualization,
Human-Computer Interaction, and Psychology communities. We focus on the
specific designs and tasks for which there is concrete evidence of performance
divergence due to individual characteristics. The purpose of this paper is to
underscore the need to consider individual differences when designing and
evaluating visualization systems and to call attention to this critical future
research direction.
\end{abstract}

\begin{IEEEkeywords}
visualization, individual differences, personality,
locus of control, cognitive abilities
\end{IEEEkeywords}

\section{Introduction}\label{Intro}
A substantial amount of work has been done to uncover the
effects that individual differences have on visualization
interactions. The implications of this research are attractive
and far-reaching; with a profound understanding of this
relationship, we could design visualizations catered to a
person’s specific needs, and infer a person’s traits from
their interactions with electronic media.  However, there is
still much work to be done, the nature of which is not always
immediately clear.

The task of surveying the role of individual differences on
visual analytics interactions is a very open-ended problem.
There are many ways to taxonomize this research, such as by
by individual trait or experimental design. However, the
distribution of research effort is uneven; there is
far more research focusing on certain individual traits,
such as locus of control, than others, such as openness.
As such, a taxonomy that simply enumerates the work done
for each trait would be limited. Instead, we use the
well-understood traits to contextualize the
state of research for all other traits
to gain clearer insignt into productive avenues for future
research.

The well understood traits that we examine are locus of control,
perceptural speed, visual working memory, verbal working
memory. We refer to these latter three traits under the
collective umbrella of "cognitive traits", because a significant
amount of research has evaluated them together. By tracking
the progression of research for these two groups, and discussing
their roles in visual analytics interactions, we can more
accurately discuss the states of other individual differences. 
The other traits we consider include the Big Five personality traits
of Openness, Conscientiousness, Extraversion, Agreeableness,
and Neuroticism. These traits are often studied in aggregate, and
can be measured reliably with the International Personality Item Pool.
However, each study surveyed has only been able to discover significances
for a subset of these traits, and our understandings of specific traits'
effects varies significanly.

Fifty seven visualization papers have been surveyed, and twelve of these
papers have been found to directly examine the relationship between visual
analytics interactions and the individual traits enumerated above. These
twelve papers, and the traits that they assessed, are shown in Figure \ref{fig}
\begin{figure*}[htbp]
    \centerline{\includegraphics[height=24em]{fig1.png}}
    \caption{Heatmap of individual differences-visualization papers and the traits
    they found significant.}
    \label{fig}
\end{figure*}

\subsection{Visual Analytics Interactions}
It is worth noting that an interaction with a visualization is not well defined. As
such, a visualization study has endless ways to measure and interpret an interaction.
However, the majority of studies surveyed have asked participants to use a visualization
to answer questions about the information portrayed, where response accuracies
serve as metrics for how "well" an interaction went. While these questions are
usually structurally similar, they have been taxonomized in multiple different ways.
One such taxonomy names questions as either procedural or inferential\cite{Green2010UsingPF}.
Procedural tasks are defined as those that can, with repetition become automatic and require
little conscious focus. Inferential tasks are those which require a person to draw conclusions
from the information available to them. Although question-based encodings enjoy the
benefits of being qualitative and consistent, they is provides a limited view of the
interactions themselves. Other methods of measurement include tracking eye movement,
time, and mouse movements as a person interacts with a visualization
\cite{Green2010UsingPF, Waldo}. Each different encoding may reveal a different perspective
with which to examine an interaction. As such, with traits for which the majority of studies
have relied on question accuracies in their assessment of interactions, it would be productive
to explore interactions with a more diverse set of encodings.



\section{Locus of Control}\label{LocusOfControl}
Locus of Control, the degree to which a person believes they
have control over their lives, has been studied closely
within the context of visualization. Green et al's 
"Using Personality Factors to Predict Interface Learning Performance"
was a seminal work in this effort. In this study,
participants were asked to complete personality inventories
to assess their degrees of locus of control, anxiety, neuroticism, 
extraversion, self-regulation, and tolerance of ambigiuty. They then
familiarized themselves with two real-world visualizations of the same dataset:
a row-based representaiton, and a pictoral, container-based representaiton.
Particpants then used these visualizations to answer a series of
procedural and inferential tasks, and their answers were analyzed with
respect to each measured personality trait. Significant correlations
were found for the traits of locus of control and anxiety, specifically that
those with an external locus of control tended to complete inferential tasks
more quickly, and those with greater anxiety measures tended to complete
iterative search tasks more quickly\cite{Green2010UsingPF}. These results suggested that the effects
of individual traits on visualization interactions varied with the manner and
purpose of these interactions. 

Since Green et al's study, Locus of control, and
individual differences in general have been studied with greater attention
to the precise nature of given visualizations and tasks. Ziemkiewicz et al's 
2011 study, "How locus of control influences compatibility with visualization style"
assessed both procedural and inferential tasks, while
examining individual differences with a more isolated assessment of layout.
This study evaluated locus of control with four visualizations of increasingly
explicit containment metaphors. These metaphors ranged from simple indentations
to mark hierarchy to series of nested boxes. It found that those with external
locus of control answered questions on the most and least explicit metaphors
faster, and performed better than those with internal locus of control in
general. There were no response time differences observed for those with
average locus of control. Search tasks also produced no significant differences
in response time across visualizations and participants\cite{Ziemkiewicz}.
This study was a progression on Green's because it isolated visualization layout
in its study of individual traits, while reinforcing the idea that the types of
tasks performed (eg procedural vs inferential) affect interactions nonuniformly.

 At this point, there was substantive evidence that locus of control impacted
 visualization interactions with respect to layout structure
and tasks performed. Ottley et al's 2012 study, "Priming Locus of Control
to Affect Performance", extended this research by examining if Locus of Control
could be "primed" before visualization interaction. People were asked
to recall situations in which they felt in control or out of control, before
participanting in a replication of Ziemkiewicz's 2011 study. Ottley et al found
that peoples' Locus of Control could be "swayed" externally or internally, which
was reflected in the replicated study's results \cite{OttleyPriming}. The idea
that individual traits, which are often thought to be static, can be influenced
carries powerful implications. As the personality-visualization relationship
becomes more well understood, whether traits can be primed to provoke specific 
interactions with a given visualization may become a relevant question.

In a return to direct application, Brown et al's 2014 study, "Finding Waldo:
Learning about Users from their Interactions" looks at the classic children's
game, Where's Waldo. After having their levels of locus of control,
extraversion, and neuroticism assessed, participants were asked to find and
identify the Waldo character as fast as possible in an image. The image was
zoomed in, and so required arrow presses to explore fully. Arrow presses,
mouse events, and task completion times were recorded. These recordings
were used to develop state based, event based, and sequence based encodings
with which to predict personality using machine learning. It was found that
all three assessed traits could be predicted from these encodings\cite{Waldo}.
This study not only introduced a new task outside of the domain of information
retrieval and analysis, but also showed that novel interaction encodings beyond
speed and accuracy had promising applications for data analysis.

While these four studies are far from an exhaustive list of work done on
locus of control, they do present a robust exploration of its relationship with
visualization. Locus of control has been evaluated with respect to a diverse
range of tasks, interaction measurement methods, and visualizations. Although
more generalized truths about the role of locus of control in visual analytics
interactions have not been uncovered, the state of locus of control research
is a good reference for other personality traits.

\section{Cognitive Abilities}
Visual working memory, verbal working memory, and perceptual speed have often
been studied with respect to visualization under the collective label
"cognitive abilities" \cite{UserAdaptive, HighlightingInterventions, ConatiLayouts}.
A primary goal of this research is to design individual-sensitive
visualizations. If a person's traits could be inferred, visualizations could
eventually be catered to that person's specific strengths and needs. Steichen
et al's study into cognitive abilities, "User-Adaptive Information Visualization:
Using Eye Gaze Data to Infer Visualization Tasks and User Cognitive Abilities"
is a useful lense into this effort\cite{UserAdaptive}. This work examines the
potential of real time eye-gaze tracking data in extracting information about
a person's traits and current activity. Study participants were presented
with either a bar or radar graph and asked to answer taxonomized questions
about the information displayed. It was shown that by tracking a participant's
eye gaze data, researchers could predict not only the type of visualization
used, but also the task at hand through observations connected with a user's
individual traits. It was found that eye gaze information at the start of an
interaction was most useful for predictions. High verbal working memory
was correlated with less time fixated on graph textual information, high
visual working memory was correlated with lower times until a first fixation,
and low perceptual speed was correlated with more time focused on graph legends.
These correlations, when observed with eye gaze data, could produce useful
real-time visualization adaptations (eg a person with low perceptual speed
could be registered in order to produce more accomodating legends.)

Adaptations are further studied with respect to cognitive abilities in
Carenini et al's "Highlighting interventions and user differences:
Informing adaptive information visualization support." This study
examined the effects of real-time adajustments to bar graphs designed to
assist in graph comprehension. It was found that these adjustments could
aid study participants in question performance. Additionally, this study
found that high perceptual speed was correlated with significantly faster
task completion, and that low verbal working memory was correlated with
worse task performance\cite{HighlightingInterventions}. A 2014 study on
cognitive abilities in the context of user layout found similar
results when measuring task completion times. It found that low verbal or visual
working memory was correlated with slower sorting times, and perceptual speed
was correlated with faster completion times across all tasks\cite{ConatiLayouts}.
However, this study also examined "high level" questions that consisted of more
general, qualitative questions, and found no significant effects of cognitive
abilities for these particular questions.
While the contexts of each of these three studies varied, the consistency of their
results lends credence to an intuivive thought on the impact of cognitive abilities:
higher levels of cognitive abilities allow more proficient interactions with visualizations.

Each of these studies had administered questions under the same task taxonomy of
"low level" tasks. This taxonomy enumerated questions as forms of information
extraction, which belonged to classes such as "Retrieve Value", "Compute Derived
Value", and "Sort"\cite{Amar}. While these classes are far from exhaustive, it is
worth noting the standardization with which these studies on cognitive abilities
have been administered. This standardization may have contributed to the strong 
consistency that has been observed for cognitive abilities. It is reasonable that
higher cognitive abilities could be particularly well suited suited to the task
of extracting information from a visualization. As such, future research efforts
may benefit from exploring novel encodings of interaction and higher level tasks.

\section{The Big Five Personality Traits}
Personality is often studied in visualization using the Five-factor model, or
Big Five personality traits. These traits consist of Openness, Conscientiousness,
Extraversion, Agreeableness, and Neuroticism. While these traits are often
evaluated together, our levels of understandings of each trait's impact on visualization
interaction varies considerably.

\subsection{Openness}\label{Openness}
Openness to experience is a trait that has not enjoyed much attention. However, it has
been shown to imply adaptability in visualization interactions. For example, a 2009 study
proposed a framework for studying visualization through the visual metaphors they
portray. It compared node-link diagrams and treemaps in their display of information
hierarchies, either as series of levels or as sets of nested containers, respectively.
Study participants were then asked to answer questions about each visualization, where
the questions were phrased with either compatible or conflicting metaphors to their
respective visualizations. The study found that this question-visualization compatibility
nearly universally impacted the accuracy of participants’ responses. The one exception
to this effect was observed in people with high levels of openness. This abscence suggests
that openness to experience allows easier adjustment to disruptions in visualization
interaction. It is worth noting that this example slightly deviates from our framework
in that it discusses the impact of openness in terms of a lack of significant observation.
As such, it is indicative of a weaker understanding of the relationship between openness
and visualization interaction. Nonetheless, this finding seems logical and carries
promising implications. Future openness research could benefit from exploring visualization
contexts that require participants to interact with more dynamic visualizations and tasks.

\subsection{Extraversion}\label{Extraversion}
Extraversion, which is often characterized by outgoing behavior, displays consistent
significance within visualization performance. In a study conducted by Sarsam et al, 
participants were divided into 2 clusters based on their personality profiles. Notably,
people with higher scores in extraversion were clustered with those with high scores in
conscientiousness, while the other cluster was marked by people with high neuroticism scores.
These clusters were then mapped to a set of UI design characteristics using association rules,
a popular method for uncovering relationships among variables. When another set of users were
asked to evaluate UI designs constructed based on these rules, the cluster they belonged to
was shown to reliably predict their preferences for UI design choices\cite{SarsamFirstLook}.
This study is especially significant not only for its evidence towards a relationship amongst
personality traits and visual analytic performance, but also for its presentation of concrete
design choices. For example, the discovered rules included specific font choices such as Verdana
vs Arial, and high vs low information density. These findings present evidence that
specific, concrete design choices can be made based on the personality profiles of users,
to increase user satisfaction. Brown et al's 2014 study also found that extraversion levels
could be predicted from the task
encodings, which suggests that different encodings which interact with extraversion could
be worthwhile to explore further\cite{Waldo}. Amongst the big five personality traits,
extraversion seems the most well studied. However, its research faces similar challenges to
those of Locus of control, in that clear, generalized conclusions regarding its interactions
with visualizations cannot yet be made. Nonetheless, because extraversion has consistently
shown significance in the tasks and interaction measurements of visualization studies, it
is reasonable to predict that future evaluations of the big five personality model will
continue to uncover effects of extraversion.

\subsection{Neuroticism}\label{Neuroticism}
Neuroticism, the tendendency to experience negative emotions, has similarly shown consistent
significance. As with extraversion, levels of neuroticism have been reliably predicted from
novel encodings of interaction\cite{Waldo}, and have been the basis for the creation of
personalized design rules\cite{SarsamFirstLook}. Ziemkiewicz's examination of layout with
respect to containment metaphors found that high levels of neuroticism were correlated with
better task performance on the layouts with most and least explicit containment metaphors
\cite{Ziemkiewicz}. Our understandings neuroticism and extraversion are at similar states
in terms of the amount of conclusions that have been reached and the diversity of interaction
contexts assessed, so we can expect a similar future progression.
%visualization studied do not put participants in situations where they are more likely to experience negative emotions, for obvious ethical concerns

\subsection{Conscientiousness}\label{Conscientiousness}
Conscientiousness which is characterized by carefulness, diligence and discipline, has generally
been thought an indicator of success, and has been shown to be positively correlated with
academic performance \cite{ImhofSpaet2013pv}. As such, it may be intuitive to hypothesize
that similar effects to those of the cognitive abilities mentioned above would be consistently
observed, since most visualization studies have studied conscientiousness with question
accuracy as the metric to capture interaction. However, these effects have not been observed.
While its effects have been examined with respect to extraversion \cite{SarsamFirstLook} to
discover concrete design choices, it has not shown substantial, isolated significance. It
therefore may be productive to assess conscientiousness in isolation, to decorrelate its
effects from those of other traits.

\subsection{Agreeableness}\label{Agreeableness}
Our understanding of the effects agreeableness, which is associated with kindness and modesty,
is markedly poor. Out of the fifty seven studies surveyed, it has neither been discussed nor
found significant. Because none of the current research has uncovered results, future
avenues of research remain unclear. While discussions of agreeableness with respect to
visual analytics interactions therefore remain speculative, it is worth considering what
apects of visualization studies examined could lead to discoveries for every trait examined
except agreeableness. Within each study, none of the tasks that participants are asked to
perform have intuitive connections to agreeableness. Moving forward, it may be productive to
design studies that examine more intuitively relevant tasks. For example, if a visualization
portrayed two conflicting beliefs sympathetically, and evaluated participant preferences
towards each belief, would a person with high measures of agreeableness exhibit a marked
openness towards both?

\section{Conclusions}
This survey is meant to reflect the current state of visualization research with respect
to individual traits. While the results observed from these traits vary significantly, there
are many avenues of future research available for each one such that we explore them with
respect to a robust evaluation of different interactions and visualizations.

\pagebreak
\bibliographystyle{ieee}
\bibliography{refs}
\vspace{12pt}

\end{document}
